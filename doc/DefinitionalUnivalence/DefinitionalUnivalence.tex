\documentclass[a4paper]{article}

\usepackage[margin=2.5cm]{geometry}
\usepackage[parfill]{parskip}
\usepackage[utf8]{inputenc}
\usepackage{titlesec}
\usepackage{xcolor}
\usepackage{hyperref}

\usepackage{amsmath,amssymb,amsfonts,amsthm}

\title{A Type Theory with Definitional Univalence}
\author{Sebastian Reichelt}

\titleformat{\paragraph}{\bfseries}{\theparagraph}{}{}

\setlength{\skip\footins}{0.5cm}

\theoremstyle{definition}
\newtheorem{definition}{Definition}[section]
\newtheorem{theorem}[definition]{Theorem}
\newtheorem{proposition}[definition]{Proposition}
\newtheorem{lemma}[definition]{Lemma}
\newtheorem{corollary}[definition]{Corollary}
\newtheorem{conjecture}[definition]{Conjecture}
\theoremstyle{remark}
\newtheorem*{remark}{Remark}
\newtheorem*{remarks}{Remarks}
\newtheorem*{example}{Example}

\newcommand{\defn}{\emph}

\let\defeq\equiv
\renewcommand{\phi}{\varphi}
\renewcommand{\equiv}{\simeq}
\renewcommand{\emptyset}{\varnothing}
\newcommand{\iso}{\cong}

\newcommand{\pathOver}[1]{=^{#1}}

\newcommand{\univVar}{\mathcal}
\newcommand{\U}{\univVar{U}}

\newcommand{\primType}{\mathbf}
\newcommand{\0}{\primType{0}}
\newcommand{\1}{\primType{1}}
\newcommand{\2}{\primType{2}}
\newcommand{\Id}{\primType{Id}}

\newcommand{\nm}{\mathsf}

\newcommand{\id}{\nm{id}}
\newcommand{\intro}{\nm{intro}}
\newcommand{\elim}{\nm{elim}}
\newcommand{\fst}{\nm{fst}}
\newcommand{\snd}{\nm{snd}}
\newcommand{\funext}{\nm{funext}}
\newcommand{\refl}{\nm{refl}}
\newcommand{\symm}{\nm{symm}}
\newcommand{\trans}{\nm{trans}}
\newcommand{\ap}{\nm{ap}}

\newcommand{\combinator}{\nm}
\newcommand{\idFun}{\combinator{I}}
\newcommand{\revAppFun}{\combinator{T}}
\newcommand{\constFun}{\combinator{K}}
\newcommand{\compFun}{\combinator{B'}}
\newcommand{\revCompFun}{\combinator{B}}
\newcommand{\swapFun}{\combinator{C}}
\newcommand{\dupFun}{\combinator{W}}
\newcommand{\revSelfAppFun}{\combinator{O}}
\newcommand{\substFun}{\combinator{S'}}
\newcommand{\revSubstFun}{\combinator{S}}
\newcommand{\constPi}{\combinator{Kd}}
\newcommand{\appPi}{\combinator{Ad}}
\newcommand{\compFunPi}{\combinator{Bd}}

\newcommand{\rightInv}{\nm{rightInv}}
\newcommand{\leftInv}{\nm{leftInv}}

\begin{document}

\maketitle

(Warning: WIP, still contains important gaps)

\section{Abstract}

We present a way to enrich an intensional type theory (containing $\Pi$ and $\Sigma$ types and
universes, but lacking induction for the moment) with identity types that \emph{definitionally}
match the behavior of identities in Homotopy Type Theory (HoTT) \cite{hottbook}.
Specifically, for every type $A$ that is an application of a primitive type former, and instances
$a,b : A,$ the expression $[a = b]$ reduces according to the extensionality principle for that type
former as known from HoTT -- in particular, univalence if $A$ is a universe. We construct a function
that implements dependent elimination for identities (i.e.\ the J rule, known as path induction in
HoTT).

Thus equality has the same properties as in HoTT except that univalence (and function
extensionality etc.) holds definitionally, preserving computation -- and improving even over cubical
type theory \cite{cubical} in this regard, where univalence is a theorem but not a definitional
equality.\footnote{The idea seems similar to ``higher observational type theory'' \cite{higher-ott},
which at the time of this writing is work in progress.}
A practical advantage is that arbitrarily complex algebraic structures can trivially be transported
along type equivalences in a theorem prover implementing this type theory, while preserving all
properties that can be defined on them. Since the resulting transport functions are reducible to a
form that avoids univalence or even function extensionality, those reduced terms can potentially
even be exported to theorem provers \emph{not} implementing this type theory.

The single trick that greatly simplifies the construction of this type theory is to replace lambda
abstractions with the $\revSubstFun,$ $\constFun,$ and $\idFun$ combinators \cite{combinators}.
Crucially, these combinators as well as all type formers and introduction and elimination rules are
well-typed (dependent) functions. Therefore we can assume that the only terms in the theory are
primitive constants and well-typed applications thereof (even though a practical implementation may
work with lambda abstractions instead or in addition). This enables us to define the behavior of
identities by induction over all primitive functions.

%Since the type theory is strongly normalizing and no normal form contains any identity types, a
%corollary is that identity types are conservative over (at least this restricted version of)
%intensional type theory.

\section{Base Theory}

The type theory we define is based on Martin-Löf dependent type theory with universes
\[\U_0 : \U_1 : \U_2 : \dots\]

The only primitive terms are the constants that we introduce below, and function application, which
we write in the form ``$f(x),$'' with ``$f(x,y,\dots)$'' as a shorthand for ``$f(x)(y)\dots$''
All other terms that are usually regarded as primitive, such as type formers and lambda
abstractions, are actually applications of primitive constants, as described in the following
subsections.

A consequence is that many definitions become recursive. In particular, typing judgments often
refer to terms that are defined later. This should not be seen as a problem because the definition
of the theory can be separated into four distinct phases:
\begin{enumerate}
  \item First, all primitive constants are introduced as untyped terms.
  \item Then reduction rules are defined on these untyped terms.
  \item Next, the primitive constants are enriched with typing judgments.
  \item Finally, we can verify the type correctness of every individual reduction rule, taking
  reduction within types into account.
\end{enumerate}
The caveat is that the reduction rules would be completely unintelligible without type information.
Therefore we choose the lesser evil of frequently referring to terms that are defined later in the
article, sometimes explicitly but often implicitly.

\subsection{Primitive Type Formers}

We assume that the theory contains the following type formers. These are quite minimalist in
order to demonstrate our basic approach; more type formers such as W types can be added if an
appropriate definition of equality exists for the types that they produce.
\begin{itemize}
  \item There are three primitive types $\0, \1, \2 : \U_n$ in any universe $\U_n.$
  \item If $\U_m$ and $\U_n$ are universes, $A : \U_m,$ and $P : A \to \U_n,$ there is a primitive
  type $\Pi(P) : \U_{\max(m,n)},$ which we often write as $\prod_{a : A} P(a).$ As usual, for
  $B : \U_n$ we define $[A \to B] :\defeq \prod_{a : A} B.$
  \item Similarly, we define $\Sigma(P) : \U_{\max(m,n)},$ often written as $\sum_{a : A} P(a),$
  and $[A \times B] :\defeq \sum_{a : A} B.$
  \item $\U_n : \U_{n+1}$ is considered a type former as well.
\end{itemize}

We can regard $\Pi$ and $\Sigma$ as (dependent) functions
\[\Pi, \Sigma : \prod_{A : \U_m} [[A \to \U_n] \to \U_{\max(m,n)}],\]
so that all type formers are just constants with specific types.

\subsection{Other Primitive Constants}

We define all introduction and elimination rules, including the three combinators which are
essentially the introduction rules for $\Pi$ types, as primitive constants with appropriate types.

\begin{itemize}
  \item $\elim_{\0,A} : \0 \to A$ for any type $A.$

  \item $\star : \1.$

  \item $0,1 : \2.$

  \item $\elim_{\2,A} : A \to A \to [\2 \to A]$ for any type $A.$

  \item $\idFun_A : A \to A$ for any type $A.$

  \item $\constFun_{AB} : B \to [A \to B]$ for all types $A$ and $B.$

  \item $\displaystyle \revSubstFun_{PQ} : \left[\prod_{a : A} \prod_{b : P(a)} Q(a,b)\right] \to \prod_{f : \Pi(P)} [A \to Q(a,f(a))]$\\[1ex]
  for $A : \U_l,$ $P : A \to \U_m,$ and $Q : \prod_{a : A} [P(a) \to \U_n].$

  $\idFun,$ $\constFun,$ and $\revSubstFun$ define how to interpret function definitions of the form
  \begin{align*}
    f : A &\to     B_a\\
        a &\mapsto b_a
  \end{align*}
  throughout this article.
  \begin{itemize}
    \item If $b_a$ is $a,$ then $f :\defeq \idFun_A.$
    \item If $b_a$ is a constant $b : B,$ i.e.\ $a$ does not appear in $b_a,$ then
    $f :\defeq \constFun_{AB}(b).$
    \item If $b_a$ is a function application $g_a(c_a)$ with $c_a : C_a$ and
    $g_a : \prod_{c : C_a} B_{ac},$ then $f :\defeq \revSubstFun_{PQ}(g,c)$ with
  \end{itemize}
  \begin{align*}
    P : A &\to     \U_m & Q : A \to C_a &\to     \U_n   & g : A &\to     \textstyle\prod_{c : C_a} B_{ac} & c : A &\to     C_a\\
        a &\mapsto C_a, &     (a,c)     &\mapsto B_{ac},&     a &\mapsto g_a,                             &     a &\mapsto c_a.
  \end{align*}

  Note that in the definition of $Q,$ we already extended the function notation to nested functions,
  which we write as functions with multiple arguments, each of which may depend on the previous.

  \item $\intro_{\Sigma(P)} : \Pi_{a : A} [P(a) \to \Sigma(P)]$ for $A : \U_m$ and $P : A \to \U_n.$

  \item $\fst_{\Sigma(P)} : \Sigma(P) \to A$ for $A : \U_m$ and $P : A \to \U_n.$

  \item $\snd_{\Sigma(P)} : \prod_{p : \Sigma(P)} P(\fst_{\Sigma(P)}(p))$ for $A : \U_m$ and $P : A \to \U_n.$
\end{itemize}

\subsection{Reduction Rules}

The reduction rules should be entirely unsurprising.
\begin{itemize}
  \item $\elim_{\2,A}(a,b,0) :\defeq a$
  \item $\elim_{\2,A}(a,b,1) :\defeq b$
  \item $\idFun_A(a) :\defeq a$
  \item $\constFun_{AB}(b,a) :\defeq b$
  \item $\revSubstFun_{PQ}(g,f,a) :\defeq g(a,f(a))$
  \item $\fst_{\Sigma(P)}(\intro_{\Sigma(P)}(a,b)) :\defeq a$
  \item $\snd_{\Sigma(P)}(\intro_{\Sigma(P)}(a,b)) :\defeq b$
\end{itemize}

\section{Identity Types}

For each $T : \U_n$ and $x,y : T$ we define a new type $[x = y] : \U_n,$ technically by introducing
a new type former but with reduction rules that provide a `definition' for it by case analysis on
$T.$
\begin{itemize}
  \item For $a,b : \1,$ $[a = b] :\defeq \1.$
  \item For $a,b : \2,$ $[a = b] :\defeq \elim_{\2, [a = b]}(\elim_{\2, [a = b]}(\1, \0, a), \elim_{\2, [a = b]}(\0, \1, a),b).$
  \item For $A : U_m,$ $P : A \to \U_n,$ and $f,g : \Pi(P),$
  $[f = g] :\defeq \prod_{a : A}\,[f(a) = g(a)].$
  \item For $A : U_m,$ $P : A \to \U_n,$ and $p,q : \Sigma(P),$
  $[p = q] :\defeq \sum_{e : \fst(p) = \fst(q)} [\snd(p) \pathOver{\ap_P(e)} \snd(q)],$
  where $\ap$ and the notation ``$\pathOver{f}$'' are defined below.
  \item For $A,B : \U_n$ we define $[A = B] :\defeq [A \equiv B],$ where ``$\equiv$'' denotes the
  type of equivalences as constructed in Homotopy Type Theory (HoTT).

  For $e : A = B,$ $a : A,$ and $b : B,$ we define $[a \pathOver{e} b] :\defeq [e(a) = b],$
  analogously to HoTT.
\end{itemize}

$\refl,$ composition, and inverse are defined by induction in the obvious way.

\subsection{Functoriality}

For $P : A \to \U_n,$ $f : \Pi(P),$ and $a,b : A,$ we define a function
\[\ap_f : \prod_{e : a = b}\:[f(a) \pathOver{\ap_P(e)} f(b)]\]
by induction over all primitive functions $f$ (including type formers) and their applications,
i.e.\ we introduce this function as a primitive constant but also define reduction rules for all
possible cases.

The type of $\ap$ refers to $\ap$ itself, but this is unproblematic because reduction rules are not
affected by types. For example, if $P$ is constant, so we have $f : A \to B$ for some $B,$ then
$\ap_P(e) \defeq \refl_B$ as defined below, and the typing judgment of $\ap_f$ reduces to
\[\ap_f : a = b \to f(a) = f(b).\]

Note that when $f$ maps to a type, $\ap_f$ maps to an equivalence between types; for now we will
only specify the forward direction in such cases, and leave the remaining data as an exercise for
the reader.

\begin{itemize}
  \item $\ap_{\Pi_A} : P = P' \to \Pi(P) = \Pi(P'),$ for $P,P' : A \to \U_n,$ is defined by
  \begin{align*}
    \ap_{\Pi_A} : \left[\prod_{a : A} [P(a) \equiv P'(a)]\right] \to \Pi(P) \to A &\to     P'(a)\\
                  (e_P,f,a)                                                       &\mapsto e_P(a,f(a)).
  \end{align*}

  \item $\ap_\Pi : \prod_{e_A : A = A'} [\Pi_A \pathOver{\ap_Q(e_A)} \Pi_{A'}],$ where
  \begin{align*}
    Q : \U_m &\to     U_{\max(m,n)+1}\\
        A    &\mapsto [[A \to \U_n] \to \U_{\max(m,n)}]
  \end{align*}
  so that
  \begin{align*}
    \ap_Q : [A \equiv A'] \to [[A \to \U_n] \to \U_{\max(m,n)}] \to [A' \to \U_n] &\to     \U_{\max(m,n)}\\
            (e_A,R,P')                                                            &\mapsto R(P' \circ e_A),
  \end{align*}
  i.e.\ $\ap_\Pi : \prod_{e_A : A \equiv A'} \prod_{P' : A' \to \U_n} [\Pi(P' \circ e_A) \equiv \Pi(P')],$
  is defined by
  \begin{align*}
    \ap_\Pi : [A \equiv A'] \to [A' \to \U_n] \to \Pi(P' \circ e_A) \to A' &\to     P'(a')\\
              (e_A,P',f,a')                                                &\mapsto \ap_{P'}(\rightInv_{e_A}(a'),f(e_A^{-1}(a'))),
  \end{align*}
  where $\rightInv$ denotes the appropriate identity in the definition of type equivalence.

  \item $\ap_{\Sigma_A} : P = P' \to \Sigma(P) = \Sigma(P'),$ for $P,P' : A \to \U_n,$ is
  defined by
  \begin{align*}
    \ap_{\Sigma_A} : \left[\prod_{a : A} [P(a) \equiv P'(a)]\right] \to \Sigma(P) &\to     \Sigma(P')\\
                     (e_P,p)                                                      &\mapsto (a,e_P(\fst(p),\snd(p))).
  \end{align*}

  \item $\ap_\Sigma : \prod_{e_A : A = A'} [\Sigma_A \pathOver{\ap_Q(e_A)} \Sigma_{A'}],$ where
  \begin{align*}
    Q : \U_m &\to     U_{\max(m,n)+1}\\
        A    &\mapsto [[A \to \U_n] \to \U_{\max(m,n)}]
  \end{align*}
  as above, is defined by
  \begin{align*}
    \ap_\Sigma : [A \equiv A'] \to [A' \to \U_n] \to \Sigma(P' \circ e_A) &\to     \Sigma(P')\\
                 (e_A,P',p)                                               &\mapsto (e_A(\fst(p)),\snd(p)).
  \end{align*}

  \item We also need to consider the type former for the identity type itself, i.e.
  \begin{align*}
    \Id_A : A \to A &\to     \U_n\\
            (a,b)   &\mapsto [a = b].
  \end{align*}
  We define its reduction rules generically, but we will need to verify that these match the
  corresponding rule when $A$ is an application of a primitive type former.

  $\ap_{\Id_A(a)} : b = b' \to [a = b] = [a = b']$ is defined by
  \begin{align*}
    \ap_{\Id_A(a)} : b = b' \to a = b &\to     a = b'\\
                     (e_b,f)          &\mapsto e_b \circ f.
  \end{align*}

  \item $\ap_{\Id_A} : a = a' \to [a = b] = [a' = b]$ is defined by
  \begin{align*}
    \ap_{\Id_A} : a = a' \to a = b &\to     a' = b\\
                  (e_a,f)          &\mapsto f \circ e_a^{-1}.
  \end{align*}

  \item $\ap_\Id : \prod_{e_A : A = A'} [\Id_A \pathOver{\ap_P(e_A)} \Id_{A'}],$ where
  \begin{align*}
    P : \U_n &\to     U_{n+1}\\
        A    &\mapsto [A \to A \to \U_n]
  \end{align*}
  so that
  \begin{align*}
    \ap_P : [A \equiv A'] \to [A \to A \to \U_n] \to A' \to A' &\to     \U_n\\
            (e_A,R,a',b')                                      &\mapsto R(e_A^{-1}(a'),e_A^{-1}(b')),
  \end{align*}
  is defined by
  \begin{align*}
    \ap_\Id : [A \equiv A'] \to e_A^{-1}(a') = e_A^{-1}(b') &\to     a' = b'\\
              (e_A,f)                                       &\mapsto (\rightInv_{e_A}(b'))^{-1} \circ \ap_{e_A}(f) \circ \rightInv_{e_A}(a').
  \end{align*}

  \item $\ap_{\idFun_A} : a = a' \to \idFun_A(a) = \idFun_A(a')$ is $\idFun_{[a = a']}.$

  \item $\ap_\idFun : \prod_{e_A : A = A'} [\idFun_A \pathOver{\ap_P(e_A)} \idFun_A'],$
  where
  \begin{align*}
    P : \U_n &\to     U_{n+1}\\
        A    &\mapsto [A \to A]
  \end{align*}
  so that
  \begin{align*}
    \ap_P : [A \equiv A'] \to [A \to A] \to A' &\to     A'\\
            (e_A,f,a')                         &\mapsto e_A(f(e_A^{-1}(a'))),
  \end{align*}
  is defined by
  \begin{align*}
    \ap_\idFun : [A \equiv A'] \to A' &\to     e_A(e_A^{-1}(a')) = a'\\
                 (e_A,a')             &\mapsto \rightInv_{e_A}(a').\\
                 %                    &\defeq  \ap_{\idFun_A}(\rightInv_{e_A}(a')),
  \end{align*}

  \item $\ap_{\constFun_{AB}(b)} : a = a' \to \constFun_{AB}(b,a) = \constFun_{AB}(b,a')$ is
  defined by
  \begin{align*}
    \ap_{\constFun_{AB}(b)} : a = a' &\to     b = b\\
                              e_a    &\mapsto \refl_b.
  \end{align*}

  \item $\ap_{\constFun_{AB}} : b = b' \to \constFun_{AB}(b) = \constFun_{AB}(b')$ is
  defined by
  \begin{align*}
    \ap_{\constFun_{AB}} : b = b' \to A &\to     b = b'\\
                           (e_b,a)      &\mapsto e_b.
  \end{align*}

  \item $\ap_{\constFun_A} : \prod_{e_B : B = B'} [\constFun_{AB} \pathOver{\ap_P(e_B)} \constFun_{AB'}],$
  where
  \begin{align*}
    P : \U_n &\to     \U_{\max(m,n)+1}\\
        B    &\mapsto [B \to [A \to B]]
  \end{align*}
  so that
  \begin{align*}
    \ap_P : [B \equiv B'] \to [B \to [A \to B]] \to B' \to A &\to     B'\\
            (e_B,f,b',a)                                     &\mapsto e_B(f(e_B^{-1}(b'),a)),
  \end{align*}
  is defined by
  \begin{align*}
    \ap_{\constFun_A} : [B \equiv B'] \to B' \to A &\to     e_B(e_B^{-1}(b')) = b'\\
                        (e_B,b',a)                 &\mapsto \rightInv_{e_B}(b')\\
                        %                          &\defeq  \ap_{\constFun_{AB'}}(\rightInv_{e_B}(b'),a).
  \end{align*}

  \item $\ap_{\constFun} : \prod_{e_A : A = A'} [\constFun_A \pathOver{\ap_P(e_A)} \constFun_{A'}],$
  where
  \begin{align*}
    P : \U_n &\to     \U_{\max(m,n)+1}\\
        A    &\mapsto \prod_{B : \U_n} [B \to [A \to B]]
  \end{align*}
  so that
  \begin{align*}
    \ap_P : [A \equiv A'] \to \left[\prod_{B : \U_n} [B \to [A \to B]]\right] \to \U_n \to B \to A' &\to     B\\
            (e_A,f,B,b,a')                                                                          &\mapsto f(b,e_A^{-1}(a')),
  \end{align*}
  is defined by
  \begin{align*}
    \ap_{\constFun} : [A \equiv A'] \to \U_n \to B \to A' &\to     b = b\\
                      (e_A,B,b,a')                        &\mapsto \refl_b\\
                      %                                   &\defeq  \ap_{\constFun_{A'B}(b)}(\rightInv_{e_A}(a')).
  \end{align*}

  \item $\ap_{\revSubstFun_{PQ}(g,f)} : a = a' \to \revSubstFun_{PQ}(g,f,a) = \revSubstFun_{PQ}(g,f,a')$
  is defined by
  \begin{align*}
    \ap_{\revSubstFun_{PQ}(g,f)} : a = a' &\to     g(a,f(a)) = g(a',f(a'))\\
                                   e_a    &\mapsto \ap_{g(a')}(\ap_f(e_a)) \circ \ap_g(e_a,f(a))\\
                                          &=       \ap_g(e_a,f(a')) \circ \ap_{g(a)}(\ap_f(e_a)).
  \end{align*}

  \item $\ap_{\revSubstFun_{PQ}(g)} : f = f' \to \revSubstFun_{PQ}(g,f) = \revSubstFun_{PQ}(g,f')$
  is defined by
  \begin{align*}
    \ap_{\revSubstFun_{PQ}(g)} : f = f' \to A &\to     g(a,f(a)) = g(a,f'(a))\\
                                 (e_f,a)      &\mapsto \ap_{g(a)}(e_f(a)).
  \end{align*}

  \item $\ap_{\revSubstFun_{PQ}} : g = g' \to \revSubstFun_{PQ}(g) = \revSubstFun_{PQ}(g')$
  is defined by
  \begin{align*}
    \ap_{\revSubstFun_{PQ}} : g = g' \to \Pi(P) \to A &\to     g(a,f(a)) = g'(a,f(a))\\
                              (e_g,f,a)               &\mapsto e_g(a,f(a)).
  \end{align*}

  \item $\ap_{\revSubstFun_P} : \prod_{e_Q : Q = Q'} [\revSubstFun_{PQ} \pathOver{\ap_R(e_Q)} \revSubstFun_{PQ'}],$
  where
  \begin{align*}
    R : \U_{\max(l,m,n+1)} &\to     \U_{\max(l,m,n)+1}\\
        Q                  &\mapsto \left[\left[\prod_{a : A} \prod_{b : P(a)} Q(a,b)\right] \to \prod_{f : \Pi(P)} [A \to Q(a,f(a))]\right]
  \end{align*}
  so that
  \begin{align*}
    \ap_R : \left[\prod_{a : A} \prod_{b : P(a)} [Q(a,b) \equiv Q'(a,b)]\right] &\to\\
            \left[\left[\prod_{a : A} \prod_{b : P(a)} Q(a,b)\right] \to \prod_{f : \Pi(P)} [A \to Q(a,f(a))]\right] &\to\\
            \left[\prod_{a : A} \prod_{b : P(a)} Q'(a,b)\right] \to \Pi(P) \to A &\to     Q'(a,f(a))\\
            (e_Q,h,g',f,a)                                                       &\mapsto e_Q(a,f(a),h(e_Q^{-1}(g'),f,a))
  \end{align*}
  (with $e_Q^{-1}(g')$ defined in the obvious way), is defined by
  \begin{align*}
    \ap_{\revSubstFun_P} : \left[\prod_{a : A} \prod_{b : P(a)} [Q(a,b) \equiv Q'(a,b)]\right] &\to\\
                           \left[\prod_{a : A} \prod_{b : P(a)} Q'(a,b)\right] \to \Pi(P) \to A &\to     e_Q(a,f(a),(e_Q(a,f(a)))^{-1}(g'(a,f(a)))) = g'(a,f(a))\\
                           (e_Q,g',f,a)                                                         &\mapsto \rightInv_{e_Q(a,f(a))}(g'(a,f(a)))\\
                                                                                                &\defeq  \ap_{\revSubstFun_{PQ'}}(\rightInv_{e_Q}(g'),f,a)
  \end{align*}
  (with $\rightInv_{e_Q}(g')$ defined in the obvious way).

  \item $\ap_{\revSubstFun_A} : \prod_{e_P : P = P'} [\revSubstFun_P \pathOver{\ap_R(e_P)} \revSubstFun_{P'}],$
  where
  \begin{align*}
    R : \U_{\max(l,m+1)} &\to     \U_{\max(l,m,n)+1}\\
        P                &\mapsto \prod_{Q : \prod_{a : A} [P(a) \to \U_n]} \left[\left[\prod_{a : A} \prod_{b : P(a)} Q(a,b)\right] \to \prod_{f : \Pi(P)} [A \to Q(a,f(a))]\right],
  \end{align*}
  is defined by
  \begin{align*}
    \ap_{\revSubstFun_A} : \left[\prod_{a : A} [P(a) \equiv P'(a)]\right] \to \left[\prod_{a : A} [P'(a) \to \U_n]\right] &\to\\
                           \left[\prod_{a : A} \prod_{b' : P'(a)} Q'(a,b')\right] \to \Pi(P') \to A &\to     g'(a,e_P(a,((e_P(a))^{-1})(f'(a)))) = g'(a,f'(a))\\
                           (e_P,Q',g',f',a)                                                         &\mapsto \ap_{g'(a)}(\rightInv_{e_P(a)}(f'(a)))\\
                                                                                                    &\defeq  \ap_{\revSubstFun_{P'Q'}(g')}(\rightInv_{e_P}(f'),a)
  \end{align*}
  (with $\rightInv_{e_P}(f')$ defined in the obvious way).

  \item $\ap_{\revSubstFun} : \prod_{e_A : A = A'} [\revSubstFun_A \pathOver{\ap_R(e_A)} \revSubstFun_{A'}],$
  where
  \begin{align*}
    R : \U_l &\to     \U_{\max(l,m,n)+1}\\
        A    &\mapsto \prod_{P : A \to \U_m}\,\prod_{Q : \prod_{a : A} [P(a) \to \U_n]} \left[\left[\prod_{a : A} \prod_{b : P(a)} Q(a,b)\right] \to \prod_{f : \Pi(P)} [A \to Q(a,f(a))]\right],
  \end{align*}
  is defined by
  \begin{align*}
    \ap_\revSubstFun : [A \equiv A'] &\to\\
                       [A' \to \U_m] \to \left[\prod_{a' : A'} [P'(a') \to \U_n]\right] &\to\\
                       \left[\prod_{a' : A'} \prod_{b' : P'(a')} Q'(a',b')\right] \to \Pi(P') \to A' &\to     g'(e_A(e_A^{-1}(a')),f'(e_A(e_A^{-1}(a')))) = g'(a',f'(a'))\\
                       (e_A,P',Q',g',f',a')                                                          &\mapsto \ap_{\revSubstFun_{P'Q'}(g',f')}(\rightInv_{e_A}(a')).
  \end{align*}

  \item $\ap$ for the remaining introduction and elimination functions is straightforward.
\end{itemize}

$\ap$ respects reflexivity and composition, thus turning functions into functors. We also need to
ensure that reduction of $\ap$ is confluent with other reductions. Informally, we can observe from
the type constraints in each case that there is almost always only one possible definition of
$\ap.$ Two exceptions are currently known:
\begin{itemize}
  \item There are two possible definitions of $\ap_{\revSubstFun_{PQ}(g,f)},$ which are equal
  because naturality is always guaranteed. This equality might not be definitional.
  \item For an equivalence $e$ between types, the equality between $(e^{-1})^{-1}$ and $e$ might
  not be definitional.
\end{itemize}

\subsection{Elimination}

For $A : \U_m,$ $a : A,$ $P : \prod_{b : A} [a = b \to \U_n],$ $b,b' : A,$ and $e_b : b = b',$ we
have
\[\ap_P(e_b) : P(b) \pathOver{\ap_S(e_b)} P(b'),\]
where
\begin{align*}
  S : A &\to     \U_{\max(m,n+1)}\\
      b &\mapsto [a = b \to \U_n].
\end{align*}
Due to the behavior of $\ap$ with respect to equality, we have
\begin{align*}
  \ap_S : b = b' \to [a = b \to \U_n] \to a = b' &\to     \U_n\\
          (e_b,P,f)                              &\mapsto P(e_b^{-1} \circ f),
\end{align*}
and therefore
\[\ap_P(e_b) : \prod_{f : b = b'} [P(b,e_b^{-1} \circ f) \equiv P(b',f)]\]
and in particular an instance of
\[P(b,\refl_b) \equiv P(b',e_b),\]
corresponding to based path induction.

\section{Conclusion}

We have constructed a type theory where identity types reduce according to their extensionality
principle in HoTT, including univalence, while at the same time behaving as expected regarding
substitution principles. A consequence is that in a potential theorem prover implementing this type
theory, equality is very well-behaved.
\begin{itemize}
  \item For each structure, which can be regarded as a nested $\Sigma$ type, equality of its
  instances is \emph{by definition} the same as isomorphism.
  \item Transporting proofs and data along an equivalence of types (in particular, along an
  isomorphism of algebraic structures), is the same as rewriting along an equality. In contrast to
  HoTT, no explicit conversion between equality and equivalence is necessary, and the resulting
  transportation is defined directly in terms of the equivalence instead of containing an
  irreducible axiom.
  \item Function extensionality holds by definition.
\end{itemize}

\section{Future Work}

We are currently verifying by computer that all reduction rules are well-typed, with respect to all
possible reductions performed on the types involved in each particular rule. To ensure consistency,
however, stronger guarantees are needed. While termination looks relatively obvious, the potential
lack of confluence (due to some necessarily arbitrary choices) is worrying.

To be useful, a type former for inductive types must be added. This should hopefully be fairly
straightforward because the correct definition of equality on inductive types is always obvious.
A less obvious case is the addition of higher inductive types (HITs), where the situation is
exactly reversed: the entire concept of HITs is footed on the complexity of their identity types.

One goal is obviously the implementation of this theory in a theorem prover. To achieve this, it
may make sense to translate the reduction rules on combinators back to lambda terms, so that
combinators are no longer needed.

A library of mathematics in such a prover would have an interesting property: Proofs and
constructions that look simple but contain complex rewrites would reduce to terms that are
``exportable,'' both to other provers and also to theories constructed in the prover itself. For
example, such terms will often still be valid when types are replaced with categories, higher
categories, or topological spaces. From another angle, this suggests that the prover could contain
a reflection mechanism that allows such structures to be treated as types, by providing appropriate
mappings.

Finally, there is certainly room for lots of minor improvements. For example, the definition of
$[a \pathOver{e} b]$ as $[e(a) = b]$ often leads to equalities that unnecessarily include the term
$e^{-1}$ on the left-hand side instead of $e$ on the right-hand side. In a practical implementation,
it would be useful to specialize the definition for the case that $a$ and $b$ are functions and $e$
originates from an equivalence on their domains, so that equality of structures reduces to the usual
definition of isomorphism instead of a type that is only equivalent but not definitionally so.

Also, the definition of equality for the type $\2$ is a bit ad-hoc. If $\2$ was regarded as a
\emph{universe} containing the \emph{types} $\0$ and $\1$ (i.e.\ classical propositions, due to the
elimination function) instead of values $0$ and $1,$ equivalence of types would yield the correct
definition. Although this just seems like a minor curiosity, there may be a useful extension of this
principle that generalizes the concept of universes, perhaps in a way that subsumes the envisioned
reflection mechanism.

\bibliographystyle{plain}
\bibliography{DefinitionalUnivalence}

\end{document}
